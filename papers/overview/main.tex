% !TeX root = ./main.tex

\documentclass{article}

\usepackage[colorlinks]{hyperref}
\hypersetup{
    colorlinks=true,
    linkcolor=blue,
    filecolor=magenta,
    urlcolor=cyan,
}
\usepackage[english]{babel}
\usepackage[symbols,nogroupskip,sort=none]{glossaries-extra}
\usepackage[utf8]{inputenc}
\usepackage{graphicx}
\usepackage{multicol}
\usepackage{siunitx}

\glsxtrnewsymbol[description={Number density}]{n}{\ensuremath{n}}
\glsxtrnewsymbol[description={Time}]{t}{\ensuremath{t}}
\glsxtrnewsymbol[description={Position}]{r}{\ensuremath{\vec{r}}}
\glsxtrnewsymbol[description={ith Length Dimension ($d_0 = x$)}]{di}{\ensuremath{d_i}}
\glsxtrnewsymbol[description={Diffusion coefficent}]{D}{\ensuremath{D}}
\glsxtrnewsymbol[description={Partial derivative}]{partial}{\ensuremath{\partial}}
\glsxtrnewsymbol[description={Vector differential operator del}]{del}{\ensuremath{\nabla}}

\begin{document}

\begin{center}
    \Large
    \textbf{Arc::Torus Code Paper and Demonstration}

    \vspace{0.4cm}
    \large
    Arc::Torus used to simulate the PDT process and predict treatment efficacy.

    \vspace{0.4cm}
    \textbf{Freddy Wordingham}

    \vspace{0.9cm}
    \textbf{Abstract}
\end{center}

This is the abstract text.

\newpage
\tableofcontents

\newpage
\printunsrtglossary[type=symbols,style=long]

\newpage
\begin{multicols}{2}

    \section{Domain}
    \subsection{Input}
    A single input manifest specifying the geometry of the surfaces and their respective materials.

    \section{Monte Carlo Radiative Transfer}

    \section{Diffusion}
    The diffusion equation is a partial differential equation used to describe the behaviour of a collection of a statistically large number of species whose collective motion of the species results from the random movement of each particle arising from brownian motion.

    The standard form of the diffusion equation is written as:

    \begin{equation} \label{eq:diff_eq}
        \frac{\partial n(\vec{r}, t)}{\partial t} = \nabla \cdot ( D(n, \vec{r}) \nabla n(\vec{r}, t))
    \end{equation}

    When the diffusion coefficient is anisotropic, the diffusion coefficient is represented as a symmetric positive definite matrix, and the diffusion rate at a given position in time is written as:

    \begin{equation} \label{eq:diff_eq}
        \frac{\partial n(\vec{r}, t)}{\partial t} = \sum_{i=0}^{2} \sum_{j=0}^{2} \sum_{k=0}^{2} \frac{\partial}{\partial x_i d_i} ( D_{i,j,k}(n, \vec{r}) )
    \end{equation}

    \section{Chemical Reactions}

\end{multicols}
\end{document}
